\documentclass[letterpaper,10pt,titlepage,draftclsnofoot,onecolumn]{IEEEtran}

\usepackage{graphicx}
\usepackage{amssymb}
\usepackage{amsmath}
\usepackage{amsthm}
\usepackage{titling}

\usepackage{alltt}
\usepackage{float}
\usepackage{color}
\usepackage{url}

\usepackage{balance}
\usepackage[TABBOTCAP, tight]{subfigure}
\usepackage{enumitem}
\usepackage{pstricks, pst-node}

\usepackage{geometry}
\geometry{textheight=8.5in, textwidth=6in}

\newcommand{\cred}[1]{{\color{red}#1}}
\newcommand{\cblue}[1]{{\color{blue}#1}}

\usepackage{hyperref}
\usepackage{geometry}

\def\name{Michael Elliott, Kiarash Teymoury, Liv Vitale}

\usepackage{fancyvrb}
\usepackage{color}
\usepackage[latin1]{inputenc}


\makeatletter
\def\PY@reset{\let\PY@it=\relax \let\PY@bf=\relax%
    \let\PY@ul=\relax \let\PY@tc=\relax%
    \let\PY@bc=\relax \let\PY@ff=\relax}
\def\PY@tok#1{\csname PY@tok@#1\endcsname}
\def\PY@toks#1+{\ifx\relax#1\empty\else%
    \PY@tok{#1}\expandafter\PY@toks\fi}
\def\PY@do#1{\PY@bc{\PY@tc{\PY@ul{%
    \PY@it{\PY@bf{\PY@ff{#1}}}}}}}
\def\PY#1#2{\PY@reset\PY@toks#1+\relax+\PY@do{#2}}

\expandafter\def\csname PY@tok@gd\endcsname{\def\PY@tc##1{\textcolor[rgb]{0.63,0.00,0.00}{##1}}}
\expandafter\def\csname PY@tok@gu\endcsname{\let\PY@bf=\textbf\def\PY@tc##1{\textcolor[rgb]{0.50,0.00,0.50}{##1}}}
\expandafter\def\csname PY@tok@gt\endcsname{\def\PY@tc##1{\textcolor[rgb]{0.00,0.25,0.82}{##1}}}
\expandafter\def\csname PY@tok@gs\endcsname{\let\PY@bf=\textbf}
\expandafter\def\csname PY@tok@gr\endcsname{\def\PY@tc##1{\textcolor[rgb]{1.00,0.00,0.00}{##1}}}
\expandafter\def\csname PY@tok@cm\endcsname{\let\PY@it=\textit\def\PY@tc##1{\textcolor[rgb]{0.25,0.50,0.50}{##1}}}
\expandafter\def\csname PY@tok@vg\endcsname{\def\PY@tc##1{\textcolor[rgb]{0.10,0.09,0.49}{##1}}}
\expandafter\def\csname PY@tok@m\endcsname{\def\PY@tc##1{\textcolor[rgb]{0.40,0.40,0.40}{##1}}}
\expandafter\def\csname PY@tok@mh\endcsname{\def\PY@tc##1{\textcolor[rgb]{0.40,0.40,0.40}{##1}}}
\expandafter\def\csname PY@tok@go\endcsname{\def\PY@tc##1{\textcolor[rgb]{0.50,0.50,0.50}{##1}}}
\expandafter\def\csname PY@tok@ge\endcsname{\let\PY@it=\textit}
\expandafter\def\csname PY@tok@vc\endcsname{\def\PY@tc##1{\textcolor[rgb]{0.10,0.09,0.49}{##1}}}
\expandafter\def\csname PY@tok@il\endcsname{\def\PY@tc##1{\textcolor[rgb]{0.40,0.40,0.40}{##1}}}
\expandafter\def\csname PY@tok@cs\endcsname{\let\PY@it=\textit\def\PY@tc##1{\textcolor[rgb]{0.25,0.50,0.50}{##1}}}
\expandafter\def\csname PY@tok@cp\endcsname{\def\PY@tc##1{\textcolor[rgb]{0.74,0.48,0.00}{##1}}}
\expandafter\def\csname PY@tok@gi\endcsname{\def\PY@tc##1{\textcolor[rgb]{0.00,0.63,0.00}{##1}}}
\expandafter\def\csname PY@tok@gh\endcsname{\let\PY@bf=\textbf\def\PY@tc##1{\textcolor[rgb]{0.00,0.00,0.50}{##1}}}
\expandafter\def\csname PY@tok@ni\endcsname{\let\PY@bf=\textbf\def\PY@tc##1{\textcolor[rgb]{0.60,0.60,0.60}{##1}}}
\expandafter\def\csname PY@tok@nl\endcsname{\def\PY@tc##1{\textcolor[rgb]{0.63,0.63,0.00}{##1}}}
\expandafter\def\csname PY@tok@nn\endcsname{\let\PY@bf=\textbf\def\PY@tc##1{\textcolor[rgb]{0.00,0.00,1.00}{##1}}}
\expandafter\def\csname PY@tok@no\endcsname{\def\PY@tc##1{\textcolor[rgb]{0.53,0.00,0.00}{##1}}}
\expandafter\def\csname PY@tok@na\endcsname{\def\PY@tc##1{\textcolor[rgb]{0.49,0.56,0.16}{##1}}}
\expandafter\def\csname PY@tok@nb\endcsname{\def\PY@tc##1{\textcolor[rgb]{0.00,0.50,0.00}{##1}}}
\expandafter\def\csname PY@tok@nc\endcsname{\let\PY@bf=\textbf\def\PY@tc##1{\textcolor[rgb]{0.00,0.00,1.00}{##1}}}
\expandafter\def\csname PY@tok@nd\endcsname{\def\PY@tc##1{\textcolor[rgb]{0.67,0.13,1.00}{##1}}}
\expandafter\def\csname PY@tok@ne\endcsname{\let\PY@bf=\textbf\def\PY@tc##1{\textcolor[rgb]{0.82,0.25,0.23}{##1}}}
\expandafter\def\csname PY@tok@nf\endcsname{\def\PY@tc##1{\textcolor[rgb]{0.00,0.00,1.00}{##1}}}
\expandafter\def\csname PY@tok@si\endcsname{\let\PY@bf=\textbf\def\PY@tc##1{\textcolor[rgb]{0.73,0.40,0.53}{##1}}}
\expandafter\def\csname PY@tok@s2\endcsname{\def\PY@tc##1{\textcolor[rgb]{0.73,0.13,0.13}{##1}}}
\expandafter\def\csname PY@tok@vi\endcsname{\def\PY@tc##1{\textcolor[rgb]{0.10,0.09,0.49}{##1}}}
\expandafter\def\csname PY@tok@nt\endcsname{\let\PY@bf=\textbf\def\PY@tc##1{\textcolor[rgb]{0.00,0.50,0.00}{##1}}}
\expandafter\def\csname PY@tok@nv\endcsname{\def\PY@tc##1{\textcolor[rgb]{0.10,0.09,0.49}{##1}}}
\expandafter\def\csname PY@tok@s1\endcsname{\def\PY@tc##1{\textcolor[rgb]{0.73,0.13,0.13}{##1}}}
\expandafter\def\csname PY@tok@sh\endcsname{\def\PY@tc##1{\textcolor[rgb]{0.73,0.13,0.13}{##1}}}
\expandafter\def\csname PY@tok@sc\endcsname{\def\PY@tc##1{\textcolor[rgb]{0.73,0.13,0.13}{##1}}}
\expandafter\def\csname PY@tok@sx\endcsname{\def\PY@tc##1{\textcolor[rgb]{0.00,0.50,0.00}{##1}}}
\expandafter\def\csname PY@tok@bp\endcsname{\def\PY@tc##1{\textcolor[rgb]{0.00,0.50,0.00}{##1}}}
\expandafter\def\csname PY@tok@c1\endcsname{\let\PY@it=\textit\def\PY@tc##1{\textcolor[rgb]{0.25,0.50,0.50}{##1}}}
\expandafter\def\csname PY@tok@kc\endcsname{\let\PY@bf=\textbf\def\PY@tc##1{\textcolor[rgb]{0.00,0.50,0.00}{##1}}}
\expandafter\def\csname PY@tok@c\endcsname{\let\PY@it=\textit\def\PY@tc##1{\textcolor[rgb]{0.25,0.50,0.50}{##1}}}
\expandafter\def\csname PY@tok@mf\endcsname{\def\PY@tc##1{\textcolor[rgb]{0.40,0.40,0.40}{##1}}}
\expandafter\def\csname PY@tok@err\endcsname{\def\PY@bc##1{\setlength{\fboxsep}{0pt}\fcolorbox[rgb]{1.00,0.00,0.00}{1,1,1}{\strut ##1}}}
\expandafter\def\csname PY@tok@kd\endcsname{\let\PY@bf=\textbf\def\PY@tc##1{\textcolor[rgb]{0.00,0.50,0.00}{##1}}}
\expandafter\def\csname PY@tok@ss\endcsname{\def\PY@tc##1{\textcolor[rgb]{0.10,0.09,0.49}{##1}}}
\expandafter\def\csname PY@tok@sr\endcsname{\def\PY@tc##1{\textcolor[rgb]{0.73,0.40,0.53}{##1}}}
\expandafter\def\csname PY@tok@mo\endcsname{\def\PY@tc##1{\textcolor[rgb]{0.40,0.40,0.40}{##1}}}
\expandafter\def\csname PY@tok@kn\endcsname{\let\PY@bf=\textbf\def\PY@tc##1{\textcolor[rgb]{0.00,0.50,0.00}{##1}}}
\expandafter\def\csname PY@tok@mi\endcsname{\def\PY@tc##1{\textcolor[rgb]{0.40,0.40,0.40}{##1}}}
\expandafter\def\csname PY@tok@gp\endcsname{\let\PY@bf=\textbf\def\PY@tc##1{\textcolor[rgb]{0.00,0.00,0.50}{##1}}}
\expandafter\def\csname PY@tok@o\endcsname{\def\PY@tc##1{\textcolor[rgb]{0.40,0.40,0.40}{##1}}}
\expandafter\def\csname PY@tok@kr\endcsname{\let\PY@bf=\textbf\def\PY@tc##1{\textcolor[rgb]{0.00,0.50,0.00}{##1}}}
\expandafter\def\csname PY@tok@s\endcsname{\def\PY@tc##1{\textcolor[rgb]{0.73,0.13,0.13}{##1}}}
\expandafter\def\csname PY@tok@kp\endcsname{\def\PY@tc##1{\textcolor[rgb]{0.00,0.50,0.00}{##1}}}
\expandafter\def\csname PY@tok@w\endcsname{\def\PY@tc##1{\textcolor[rgb]{0.73,0.73,0.73}{##1}}}
\expandafter\def\csname PY@tok@kt\endcsname{\def\PY@tc##1{\textcolor[rgb]{0.69,0.00,0.25}{##1}}}
\expandafter\def\csname PY@tok@ow\endcsname{\let\PY@bf=\textbf\def\PY@tc##1{\textcolor[rgb]{0.67,0.13,1.00}{##1}}}
\expandafter\def\csname PY@tok@sb\endcsname{\def\PY@tc##1{\textcolor[rgb]{0.73,0.13,0.13}{##1}}}
\expandafter\def\csname PY@tok@k\endcsname{\let\PY@bf=\textbf\def\PY@tc##1{\textcolor[rgb]{0.00,0.50,0.00}{##1}}}
\expandafter\def\csname PY@tok@se\endcsname{\let\PY@bf=\textbf\def\PY@tc##1{\textcolor[rgb]{0.73,0.40,0.13}{##1}}}
\expandafter\def\csname PY@tok@sd\endcsname{\let\PY@it=\textit\def\PY@tc##1{\textcolor[rgb]{0.73,0.13,0.13}{##1}}}

\def\PYZbs{\char`\\}
\def\PYZus{\char`\_}
\def\PYZob{\char`\{}
\def\PYZcb{\char`\}}
\def\PYZca{\char`\^}
\def\PYZam{\char`\&}
\def\PYZlt{\char`\<}
\def\PYZgt{\char`\>}
\def\PYZsh{\char`\#}
\def\PYZpc{\char`\%}
\def\PYZdl{\char`\$}
\def\PYZti{\char`\~}
% for compatibility with earlier versions
\def\PYZat{@}
\def\PYZlb{[}
\def\PYZrb{]}
\makeatother

\hypersetup{
  colorlinks = true,
  urlcolor = black,
  pdfauthor = {\name},
  pdfkeywords = {cs444 ''operating systems'' files  filesystem I/O},
  pdftitle = {CS 444 Homework 1},
  pdfsubject = {CS 444 Homework 1},
  pdfpagemode = UseNone
}

\title{CS 444 Homework 1}
\author{Michael Elliott, Kiarash Teymoury, Liv Vitale}

\begin{document}

\begin{titlingpage}
\maketitle
\begin{abstract}
This document details our experience compiling the linux kernel and booting it via QEMU as well as our work on the concurrency assignment. Additionally, our full commit log and work schedules are included.
\end{abstract}
\end{titlingpage}

\section{Kernel Compilation Commands}
\textdollar{} cd /scratch/spring2017
\\\textdollar{} mkdir 13-04
\\\textdollar{} cd 13-04
\\\textdollar{} screen -S os
\\\textdollar{} git clone git://git.yoctoproject.org/linux-yocto-3.14
\\\textdollar{} cd linux-yocto-3.14
\\\textdollar{} git checkout v3.14.26
\\\textdollar{} cd ..
\\\textdollar{} cp /scratch/opt/environment-setup-i586-poky-linux.csh .
\\\textdollar{} source environment-setup-i586-poky-linux.csh
\\\textdollar{} cd linux-yocto-3.14
\\\textdollar{} cp /scratch/spring2017/files/config-3.14.26-yocto-qemu .config
\\\textdollar{} sed -i 's/\^{}CONFIG\_LOCALVERSION=''[\^{}'']*''\textdollar{}/CONFIG\_LOCALVERSION=''-13-04-hw1''/' .config
\\\textdollar{} make -j4
\\\textdollar{} cd -
\\\textdollar{} qemu-system-i386 -gdb tcp::5634 -S -nographic -kernel ``linux-yocto-3.14/arch/x86/boot/bzImage'' -drive file=''core-image-lsb-sdk-qemux86.ext3'',if=virtio -enable-kvm -net none -usb -localtime --no-reboot --append ``root=/dev/vda rw console=ttyS0 debug''
C-a c
\\\textdollar{} echo ``target remote localhost:5634\textbackslash{}ncontinue'' \textgreater{} .gdbinit
\\\textdollar{} echo ``add-auto-load-safe-path /scratch/spring2017/13-04/.gdbinit'' \textgreater{}\textgreater{} \textasciitilde{}/.gdbinit
\\\textdollar{} \textdollar{}GDB
C-a 0
\\qemux86 login: root
\\\# uname -a
\\\# reboot



\section{QEMU Flags}
\begin{tabular}{l | c}
Flag & Explanation \\
\hline
-gdb [port] & Defines the port to which the GNU debugger should wait for a connection. \\
-S & Sets the CPU to not start at VM startup. \\
-nographic & Disables graphical output, effectively making QEMU run as a console application. \\
-kernel [file] & Sets the linux kernel to boot to. \\
-drive [options] & Mounts a virtual drive. \\
-enable-kvm & Enables KVM virtualization support, allowing for virtualizing a kernel. \\
-net [options] & Creates a virtual network adapter (in our case, none). \\
-usb & Enables USB support. \\
-no-reboot & Exits instead of rebooting. \\
-append [cmdline] & Specifies a command line to use as the kernel command line. \\
\end{tabular}

\section{Concurrency Assignment}
\begin{enumerate}
\item \textbf{What do you think the main point of this assignment is?}

The purpose of this assignment was mainly to start thinking in parallel \par

\item \textbf{How did you personally approach the problem? Design decisions, algorithm, etc}

  The first design question was how to organize the items in the buffer in a way in which they could be easily accessed by multiple threads without interference. The solution we came up with was to use a simple locking queue data structure as the buffer. The queue is a singly linked list with O(1) access to the first and last nodes as well as O(1) add and pop operations. Since the intermediate nodes in the queue would never be accessed, we did not have to worry about the O(n) time it would take to reach them. The queue also has a mutex property that can be locked and unlocked. The operations that can be performed on the queue are as follows: an empty check, a full check, add last, pop first, lock, and unlock. This way multiple threads can modify the queue by first locking it, then adding to or popping from the queue, then unlocking it again. When another thread tries to lock a locked queue, it blocks until the queue becomes available. \par
  The second design question was how to structure the thread routines so they could effectively do work on the buffer. The producer thread routine blocks until the buffer is not full, locks the buffer, adds a new item to the queue, unlocks the buffer, and then sleeps for 3-7 seconds. The consumer thread works quite similarly: it blocks until the buffer is not empty, locks the buffer, pops the first item from the queue, unlocks the buffer, and then performs the work on the item. \par
  The program allows the user to specify the number of producer and consumer threads as command line arguments. As long as the arguments are positive integers, the program will spawn the specified number of threads of each type. This allows the user to tailor the number of threads to their system for maximum efficiency. \par

\item \textbf{How did you ensure your solution was correct? Testing details, for instance.}

We unit tested each function to make sure they worked properly on their own. We also had the final program print incremental messages about what it was doing when in order to make sure everything fit together properly. \par

\item \textbf{What did you learn?}

We learned a lot about how LaTeX works as well as the concept of parallel thinking. \par

\end{enumerate}

\section{Version Control Log}
\begin{tabular}{l l l}\textbf{Detail} & \textbf{Author} & \textbf{Description}\\\hline
\href{https://github.com/elliomic/CS-444/commit/495c3654bd7736ea43c6a88eb59743eab5e64695}{495c365} & Michael Elliott & Added Mersenne Twister header\\\hline
\href{https://github.com/elliomic/CS-444/commit/93039f88419ac754dcbd9f5e86c4acc2ae075c92}{93039f8} & Michael Elliott & Added simple queue data structure\\\hline
\href{https://github.com/elliomic/CS-444/commit/448f904d4299eed4f7290557a7dd2ed03a731583}{448f904} & Michael Elliott & Added actual header file for Mersenne Twister\\\hline
\href{https://github.com/elliomic/CS-444/commit/a569e9c76e5f3b7d77a61959641102f75df9826e}{a569e9c} & Michael Elliott & Added size constraints to the queue data structure\\\hline
\href{https://github.com/elliomic/CS-444/commit/793d488cb52123a48e8d345293ce4359e2c301c1}{793d488} & Michael Elliott & Added mutex property to queue data structure\\\hline
\href{https://github.com/elliomic/CS-444/commit/acb8c0e69e610e33efc6d0764caa3ccc66a663d9}{acb8c0e} & Michael Elliott & Added function to check if a queue is full\\\hline
\href{https://github.com/elliomic/CS-444/commit/503f004502547d608cf63b7cd5b60aad8b3ab3f7}{503f004} & Michael Elliott & Added lock and unlock functions for the queue data structure\\\hline
\href{https://github.com/elliomic/CS-444/commit/66f8c69f4b2b4122718aad4f72a9eeadd9e06545}{66f8c69} & Michael Elliott & Created first concurrency program. Added functions for random numbers and added thread functions\\\hline
\href{https://github.com/elliomic/CS-444/commit/4ad6d5f2f7a954616d7a38cb8cecd9c26fc3c5f3}{4ad6d5f} & Michael Elliott & Embedded queue lock and unlock into add and pop functions\\\hline
\href{https://github.com/elliomic/CS-444/commit/9952d2b8061cbde3abc5e0b6fb4430c04aaa28e7}{9952d2b} & Michael Elliott & Removed manual locking and unlocking of queue from thread functions\\\hline
\href{https://github.com/elliomic/CS-444/commit/fbc9ff6fcc38d42815a70a9a0efff13171a768ac}{fbc9ff6} & Michael Elliott & Added Makefile. Added empty queue checking to the pop function. Fixed rand range. Moved queue locking from internally managed to managed in the thread routines. Added the ability to choose the number of threads spawned through command line arguments.\\\hline
\href{https://github.com/elliomic/CS-444/commit/4538b939bed2cc80ab0b8940c419877d51f693bc}{4538b93} & Michael Elliott & Fixed cpuid call\\\hline
\href{https://github.com/elliomic/CS-444/commit/d976f657bc72165f444aac68eeecfdda2d2e1afe}{d976f65} & Michael Elliott & Removed use of variable length arrays. Added error checking for negative input\\\hline
\href{https://github.com/elliomic/CS-444/commit/615634bdd2be122c6401c6c7ccf23e457ca09c84}{615634b} & Liv Vitale & Added LaTeX document\\\hline
\href{https://github.com/elliomic/CS-444/commit/3395b2e4d682283838e8cb97063e5517bb5b5d30}{3395b2e} & Liv Vitale & Merge branch 'master' of https://github.com/elliomic/CS-444\\\hline
\href{https://github.com/elliomic/CS-444/commit/1182169eca1f04759352581b206d7715f24499ca}{1182169} & elliomic & Delete synchro1\\\hline
\href{https://github.com/elliomic/CS-444/commit/9a46c3c9bab82e9618cddc3e1ddca4f4e3ae0765}{9a46c3c} & elliomic & Return to using gcc\\\hline
\href{https://github.com/elliomic/CS-444/commit/ca99537566dcc4c0aa6cac3e045d5fefad09dd46}{ca99537} & Liv Vitale & Added title, IEEEtran package\\\hline
\href{https://github.com/elliomic/CS-444/commit/795c3046ccb4c7bab06b6d779d74d179c68776d6}{795c304} & Liv Vitale & Added sections\\\hline
\href{https://github.com/elliomic/CS-444/commit/b8ac2a387ab9dca744d2063979ae316506860c55}{b8ac2a3} & Michael Elliott & Added command log for compiling and running Linux kernel\\\hline
\href{https://github.com/elliomic/CS-444/commit/a7ed1820ead89a461627e47e171f3de86208199c}{a7ed182} & Liv Vitale & Added script for generating .tex from git logs\\\hline
\href{https://github.com/elliomic/CS-444/commit/1e284e3a9848cf7b3c281c86e4d0bf738b2ba78e}{1e284e3} & teymourk & Added Concurrency questions\\\hline
\href{https://github.com/elliomic/CS-444/commit/cc6a7bce3e24bd2b737129c50a82f9e9217d1553}{cc6a7bc} & Liv Vitale & Escaped special chars\\\hline
\href{https://github.com/elliomic/CS-444/commit/f660d8dc7b162be993a0823f90730740dd68b9c8}{f660d8d} & Liv Vitale & Escaped characters\\\hline
\href{https://github.com/elliomic/CS-444/commit/6847089ba9f576200b15793b9ade667cf424751b}{6847089} & Liv Vitale & Added work log\\\hline
\href{https://github.com/elliomic/CS-444/commit/622d36cfa52d8585b434b9eea93ceb2294168507}{622d36c} & Michael Elliott & Removed intermediate steps of LaTeX compilation from source control. Moved everything into one flat directory. Started combining LaTeX and C makefiles.\\\hline
\href{https://github.com/elliomic/CS-444/commit/9299aa4a74db60c7cd19fd008c48fe449511c4ec}{9299aa4} & Michael Elliott & Finished Makefile\\\hline
\href{https://github.com/elliomic/CS-444/commit/5eac56f022bbc2c253f71597fe0543cf030e1ed8}{5eac56f} & Michael Elliott & Answered the design question\\\hline
\href{https://github.com/elliomic/CS-444/commit/b44b2e9e6f66bb027e0bf7568dd80d9d78853b6f}{b44b2e9} & teymourk & Finished concurrency questions\\\hline
\href{https://github.com/elliomic/CS-444/commit/27862c5569e1ac9946bc771b8cad001cdac28207}{27862c5} & Michael Elliott & Updated LaTeX git log\\\hline
\href{https://github.com/elliomic/CS-444/commit/96629b00a2b4900224980df24f5aff5be755b92d}{96629b0} & Liv Vitale & Added abstract, QEMU\\\hline
\href{https://github.com/elliomic/CS-444/commit/c9c143e3f2a20924cf39c0ec2b6185bba293ac13}{c9c143e} & Liv Vitale & Merge branch 'master' of https://github.com/elliomic/CS-444\\\hline\end{tabular}



\section{Work Log}
\begin{tabular}{l | c | r}
Date + Time & Item(s) worked on \\
\hline
4/19/17: 6:00pm - 7:00pm & Compiled kernel + QEMU \\
4/20/17: 4:00pm - 11:00pm & Concurrency assignment 1 \\
4/21/17: 5:00pm - 6:00pm & Concurrency assignment 1 bugfixes \\
4/21/17: 6:00pm - 11:30pm & LaTeX writeup \\
\end{tabular}

\nocite{*}

\bibliography{test}
\bibliographystyle{plain}

\end{document}
