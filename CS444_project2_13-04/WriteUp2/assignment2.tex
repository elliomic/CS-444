{\bf }\documentclass[letterpaper,10pt,titlepage,draftclsnofoot,onecolumn]{IEEEtran}

\usepackage{graphicx}
\usepackage{amssymb}
\usepackage{amsmath}
\usepackage{amsthm}
\usepackage{titling}

\usepackage{alltt}
\usepackage{float}
\usepackage{color}
\usepackage{url}

\usepackage{balance}
\usepackage[TABBOTCAP, tight]{subfigure}
\usepackage{enumitem}
\usepackage{pstricks, pst-node}

\usepackage{geometry}
\geometry{textheight=8.5in, textwidth=6in}

\newcommand{\cred}[1]{{\color{red}#1}}
\newcommand{\cblue}[1]{{\color{blue}#1}}

\usepackage{hyperref}
\usepackage{geometry}

\def\name{Michael Elliott, Kiarash Teymoury, Liv Vitale}

\input{pygments.tex}

\hypersetup{
  colorlinks = true,
  urlcolor = black,
  pdfauthor = {\name},
  pdfkeywords = {cs444 ''operating systems'' files  filesystem I/O},
  pdftitle = {CS 444 Homework 2},
  pdfsubject = {CS 444 Homework 2},
  pdfpagemode = UseNone
}

\title{CS 444 Homework 2}
\author{Michael Elliott, Kiarash Teymoury, Liv Vitale}

\begin{document}

\section{Design Plan}


\section{Version Control Log}

\begin{tabular}{l l l}\textbf{Detail} & \textbf{Author} & \textbf{Description}\\\hline
\href{https://github.com/elliomic/CS-444/commit/3960007dcd2e09104c3b8163b326ba635a00fe25}{3960007} & Michael Elliott & Initial commit for project2. Concurrency program is working\\\hline
\href{https://github.com/elliomic/CS-444/commit/ff5ae8022f9a2cb63b4174945bec1f7ec3d67d77}{ff5ae80} & Kiarash Teymoury & Added Latex Template\\\hline
\href{https://github.com/elliomic/CS-444/commit/2550a5434d0e99148462ae4c27ee47d6649e1f84}{2550a54} & Kiarash Teymoury & Updated Latex Template\\\hline
\href{https://github.com/elliomic/CS-444/commit/f6ca797f7076ba4a3f300e6e777f1ff3dd782a0b}{f6ca797} & Michael Elliott & Added Kernel patch\\\hline
\href{https://github.com/elliomic/CS-444/commit/eba8729a916481d064ded4e5d0a9bc6aca866996}{eba8729} & Michael Elliott & Merge branch 'master' of https://github.com/elliomic/CS-444\\\hline
\href{https://github.com/elliomic/CS-444/commit/75de20f36af8f4e47207e25020e866469c3df88a}{75de20f} & Michael Elliott & Removed rendered LaTeX files\\\hline
\href{https://github.com/elliomic/CS-444/commit/620aaa92bd40125290c7e8fea12d804613227c1e}{620aaa9} & Kiarash Teymoury & Merge branch 'master' of https://github.com/elliomic/CS-444\\\hline
\hline\end{tabular}


\section{Work Log}


\section{Concurrency Assignment}
\begin{enumerate}
\item \textbf{What do you think the main point of this assignment is?}

The main point of this assignment was to get a better understanding of how the I/O scheduler works and is implemented using the SSTF algorithm. \par

\item \textbf{How did you personally approach the problem? Design decisions, algorithm, etc}

We decided to make a data structure for all of the data needed to process each philosopher. The struct contains the pthread for that philosopher, a pointer to that philosopher's left and right ``forks'' (mutexes), and the pointer to the global semaphore. The function run by each thread takes the pointer to the philosopher as an argument. It then enters an endless loop in which the thread ``thinks'' (sleeps for 1-20s), decrements the global semaphore, locks the mutexes, ``eats'' (sleeps for 2-9s), unlocks the mutexes, and increments the semaphore. The initial value of the semaphore is equal to the number of philosophers minus 1. This way only n-1 philosophers can attempt to grab forks at any given time, meaning that at least one is guaranteed to grab both forks. This also prevents starvation because as soon as the philosopher holding both forks finishes, the semaphore is incremented and the waiting philosopher can begin the process of picking up his forks and eating.

\item \textbf{How did you ensure your solution was correct? Testing details, for instance.}

We ensured our solution is correct by printing out messages for each event. These messages showed when each philosopher was thinking and eating, and which forks were held by which philosophers. This way we were able to ensure that each philosopher got a turn as well as ensure that multiple philosophers could eat simultaneously.

\item \textbf{What did you learn?}

Answer goes here.

\end{enumerate}

\nocite{*}

\bibliography{test}
\bibliographystyle{plain}

\end{document}
