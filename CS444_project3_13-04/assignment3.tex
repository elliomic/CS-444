\documentclass[letterpaper,10pt,titlepage,draftclsnofoot,onecolumn]{IEEEtran}

\usepackage{graphicx}
\usepackage{amssymb}
\usepackage{amsmath}
\usepackage{amsthm}
\usepackage{titling}

\usepackage{alltt}
\usepackage{float}
\usepackage{color}
\usepackage{url}

\usepackage{balance}
\usepackage[TABBOTCAP, tight]{subfigure}
\usepackage{enumitem}
\usepackage{pstricks, pst-node}

\usepackage{geometry}
\geometry{textheight=8.5in, textwidth=6in}

\newcommand{\cred}[1]{{\color{red}#1}}
\newcommand{\cblue}[1]{{\color{blue}#1}}

\usepackage{hyperref}
\usepackage{geometry}

\def\name{Michael Elliott, Kiarash Teymoury, Liv Vitale}

\input{pygments.tex}

\hypersetup{
  colorlinks = true,
  urlcolor = black,
  pdfauthor = {\name},
  pdfkeywords = {cs444 ''operating systems'' files  filesystem I/O},
  pdftitle = {CS 444 Homework 2},
  pdfsubject = {CS 444 Homework 2},
  pdfpagemode = UseNone
}

\title{CS 444 Homework 2}
\author{Michael Elliott, Kiarash Teymoury, Liv Vitale}

\begin{document}

\section{Design Plan}


\section{Version Control Log}

\begin{tabular}{l l l}\textbf{Detail} & \textbf{Author} & \textbf{Description}\\\hline
\href{https://github.com/elliomic/CS-444/commit/cc528e366648a00c65aba64a9f1f2ba0f2559acb}{cc528e3} & Commit for concurrency 3\\\hline
\href{https://github.com/elliomic/CS-444/commit/978270e33710eee398a8e62dc35cba54fb573f49}{978270e} & Added makefiles and stuff for LaTeX\\\hline
\href{https://github.com/elliomic/CS-444/commit/a71a21724fcc71dec65b67e55cf5452a8fbcd1cc}{a71a217} & Added code for a RAM block device. Still need to add code for encryption\\\hline
\href{https://github.com/elliomic/CS-444/commit/c977235b57d99b5232287fd15100ed4ce500282a}{c977235} & Now have a working ramdisk useable as a filesystem. Need to do encryption now.\\\hline
\hline\end{tabular}


\section{Work Log}
\begin{tabular}{l | c | r}
Date + Time & Item(s) worked on \\
\hline
5/18/17: 11:00am - 12:00pm & Started Concurrency program \\
5/19/17: 3:00pm - 4:00pm & Finished Concurrency program \\
5/22/17: 6:00pm - 12:00am & Worked on the Kernel \\
\end{tabular}


\section{Concurrency Assignment}
\begin{enumerate}
\item \textbf{What do you think the main point of this assignment is?}

One of the points of this was to make you think parallel and was pretty much like the first concurency.\par

\item \textbf{How did you personally approach the problem? Design decisions, algorithm, etc}

We went based off of our first concurency problem and and then modified the producer and consumers to be searchers, inserters and deleters \par

\item \textbf{How did you ensure your solution was correct? Testing details, for instance.}

We put mutexes and print statements at the right times to ensure that solution was correct.\par

\item \textbf{What did you learn?}

We learned more about deleting and inserting with mutex and better ways to approach the problem.\par

\end{enumerate}

\nocite{*}

\bibliography{test}
\bibliographystyle{plain}

\end{document}
